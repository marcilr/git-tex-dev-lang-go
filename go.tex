%% -*- Mode: LaTeX -*-
%%
%% go.tex
%% Created Fri Aug  5 15:28:28 AKDT 2016
%% Copyright (C) 2016 by Raymond E. Marcil <marcilr@gmail.com>
%% 
%% Go - Everything You Wanted to Know About Go and Didn't Know to Ask
%%

  %%
%%%%%% Preamble.
  %%

%% Specify DVIPS driver used by things like hyperref
\documentclass[12pt,letterpaper,dvips]{article}

%% rcs is the package to display cvs revision info.
%%\usepackage{rcs}
\usepackage{fullpage}
\usepackage{fancyvrb} 
\usepackage{graphicx}
\usepackage{figsize}
\usepackage{calc}

%%
%% enumitem – Control layout of itemize, enumerate, description
%% https://www.ctan.org/pkg/enumitem
%%
%% Allows for use of \bgein{itemize}[leftmargin=0pt] 
%% to lists with 0 left margin.
%%
%% Itemize left margin
%% http://tex.stackexchange.com/questions/170525/itemize-left-margin
%% 
\usepackage{enumitem}%     http://ctan.org/pkg/enumitem


%% caption package for use in justifying table or figure captions
\usepackage{caption}

\usepackage{xspace}
\usepackage{booktabs}
\usepackage[first,bottomafter]{draftcopy}
\usepackage[numbib]{tocbibind}

\usepackage{amssymb}              %% AMS Symbols, used for \checkmark
\usepackage{multicol}

%%
%% Extract SVN metadata for use elsewhere.
%% This information has:
%% o the filename
%% o the revision number
%% o the date and time of the last Subversion co command
%% o name of the user who has done the action
%%
\usepackage{svninfo}
\svnInfo $Id: template.tex 52 2013-02-04 22:32:54Z marcilr $

%%
%% Hyperref package for embedding URLs for clickable links in PDFs, 
%% also specify PDF attributes here.
%%
%% The pdfborder={0 0 0} is what ellimated the blue box around the url
%% displayed by \href{}{}.
%%
%% The command pdfborder={0 0 1} would display a box with thickness of 1 pt.
%%
%% Hypertext marks in LATEX: a manual for hyperref
%% by Sebastian Rahtz and Heiko Oberdiek - November 2012
%% http://ctan.org/pkg/hyperref 
%% http://mirror.hmc.edu/ctan/macros/latex/contrib/hyperref/doc/manual.html
%%
\usepackage[
colorlinks,
linkcolor=blue,
%%colorlinks=false,
hyperindex=false,
urlcolor=blue,
pdfborder={0 0 0},
pdfauthor={Raymond E. Marcil},
pdftitle={LaTeX Template with Examples},
pdfcreator={ps2pdf},
pdfsubject={LaTeX Examples},
pdfkeywords={LaTeX Examples}
]{hyperref}


%%
%% Extract RCS metadata for use elsewhere.
%% Jason figured this out, very cool.
%%
%%\RCS $Revision: 1.53 $
%%\RCS $Date: 2006/06/26 21:04:55 $


  %%
%%%%%% Customization.
  %%

% On letter paper with 10pt font the Verbatim environment has 65 columns.
% With 12pt font the environment has 62 columns.  Exceeding this will exceed
% the frame and will look ugly.  YHBW.  HAND.
\RecustomVerbatimEnvironment{Verbatim}{Verbatim}{frame=single}

\renewenvironment{description}
                 {\list{}{\labelwidth 0pt \iteminden-\leftmargin
                          \let\labelsep\hsize
                          \let\makelabel\descriptionlabel}}
                 {\endlist}
\renewcommand*\descriptionlabel[1]{\hspace\labelsep\sffamily\bfseries #1}


  %%
%%%%%% Commands.
  %%

\newcommand{\FIXME}[1]{\textsf{[FIXME: #1]}}
\newcommand{\cmd}[1]{\texttt{#1}}


%% Squeeze space above/below captions
\setlength{\abovecaptionskip}{4pt}   % 0.5cm as an example
\setlength{\belowcaptionskip}{4pt}   % 0.5cm as an example


%% Tex really adds a lot of whitespace to itemized 
%% lists so define a new command itemize* with a 
%% lot less whitespace.  Found this in the British
%% Tex faq.
\newenvironment{itemize*}%
  {\begin{itemize}%
    \setlength{\itemsep}{0pt}%
    \setlength{\parsep}{0pt}}%
  {\end{itemize}}

  
%%
%% Tex really adds a lot of whitespace to itemized 
%% lists so define a new command itemize* with a 
%% lot less whitespace.  Found this in the British
%% Tex faq.
%%
%% Tue Jun 23 13:22:04 AKDT 2015
%% =============================
%% Added [leftmargin=0.0mm] to set the left margin=0
%% This requires use of the enumitem package:
%%   \usepackage{enumitem}%     http://ctan.org/pkg/enumitem
%%
%% Itemize left margin
%% http://tex.stackexchange.com/questions/170525/itemize-left-margin
%%
\newenvironment{itemizenoleft*}%
  {\begin{itemize}[leftmargin=15.0pt]%
    \setlength{\itemsep}{0pt}%
    \setlength{\parsep}{0pt}}%
  {\end{itemize}}
  

%%
%% Tex really adds a lot of whitespace to itemized 
%% lists so define a new command enumerate* with a 
%% lot less whitespace.  Created using itemize*
%% pattern.  
%%
  \newenvironment{enumerate*}%
  {\begin{enumerate}%
    \setlength{\itemsep}{0pt}%
    \setlength{\parsep}{0pt}}%
  {\end{enumerate}}


%%
%% Tex really adds a lot of whitespace to itemized 
%% lists so define a new command enumerate* with a 
%% lot less whitespace.  Created using itemize*
%% pattern.  
%%
%% Tue Jun 23 13:22:04 AKDT 2015
%% =============================
%% Added [leftmargin=0.0mm] to set the left margin=0
%% This requires use of the enumitem package:
%%   \usepackage{enumitem}%     http://ctan.org/pkg/enumitem
%%
%% Itemize left margin
%% http://tex.stackexchange.com/questions/170525/itemize-left-margin
%%
\newenvironment{enumeratenoleft*}%
  {\begin{enumerate}[leftmargin=0.0mm]%
    \setlength{\itemsep}{0pt}%
    \setlength{\parsep}{0pt}}%
  {\end{enumerate}}


%% Squeeze space
\renewcommand\floatpagefraction{.9}
\renewcommand\topfraction{.9}
\renewcommand\bottomfraction{.9}
\renewcommand\textfraction{.1}   
\setcounter{totalnumber}{50}
\setcounter{topnumber}{50}
\setcounter{bottomnumber}{50}


  %%
%%%%%% Document.
  %%

\title{Go\\
  \normalsize{Everything You Wanted to Know About Go\\
  and\\
  Didn't Know to Ask}}

\author{Raymond E. Marcil\\
        \texttt{$<$marcilr@gmail.com$>$}
}

% Display subversion revision and date under author on 1st page.
\date{Revision \svnInfoRevision
      \hspace{2pt}
      (\svnInfoLongDate)}


% Set date to RCS revision date
%%\date{Revision \RCSRevision
%%      \hspace{2pt}
%%      (\RCSDate)}


%%
%% Display SVN (subversion) version data at top right of 1st page,
%% This may be preferable to underneath the author.
%%
%%\rhead{Revision \svnInfoRevision\\
%%\svnInfoLongDate}


\begin{document}

\maketitle

\begin{abstract}
  Working at GCI Network Services, OSS I needed a location for Go
  documentation. Hence this document.

\end{abstract}

\vspace{2.0in}

%% Draw DNR logo and address at bottom of page
%%\begin{figure}[h]
%%        \hspace{0.32in}
%%        \SetFigLayout{1}{1}
%%        \begin{minipage}[b]{0.16\figwidth}
%%                \includegraphics[width=\textwidth]{dnr_bwlogo.eps}
%%        \end{minipage}
%%        \hspace{5pt}
%%        \begin{minipage}[b]{\figwidth}
%%                \bf{Alaska Department of Natural Resources}\\
%%                \small{\sf{Division of Support Services\\
%%                Land Records Information Section\\
%%                550 W. 7th Ave. Suite 706\\
%%                Anchorage, Alaska 99501}}
%%        \end{minipage}
%%\end{figure}

\newpage
\tableofcontents

\newpage
\listoffigures
\listoftables


%% =============== List of Abbreviations ===============
%% =============== List of Abbreviations ===============
\newpage
\setcounter{secnumdepth}{0}
\section{List of Definitions and Abbreviations}
\begin{itemize*}
  \item{\begin{bf}MOA\end{bf}} - Municipality of Anchorage

\end{itemize*}


%% ====================== Introduction ===========================
%% ====================== Introduction ===========================
\newpage
\section{Introduction}
Introduction to the \LaTeX\ Template with Examples.


%% ======================= Comments =======================
%% ======================= Comments =======================
\subsection{Comments}
\begin{center}
\framebox{
\begin{minipage}[t]{0.9\textwidth}
\cmd{COMMENTS} Comment --- \emph{Sean Weems, Spring 2003}\\
We should get the \cmd{COMMENTS} column searchable via the 
landrecords application before we do much anything else -- shouldn't
be too hard.
\end{minipage}
}
\end{center}

\begin{center}
\framebox{
\begin{minipage}[t]{0.9\textwidth}
\emph{Errata: Plats spanning multiple sections}\\
A few anomalies can be observed in the \cmd{AKPLATS}
table. Specifically plats exist that span multiple sections. 
Since the table only has a single column, \cmd{SCODE}, 
that accepts a single section code, SGU (Status
Graphics Unit) has handled this problem by entering multiple 
rows in the table, each with a different section that point to
the same plat or file. Multiple section plats are indicated by  setting 
the \cmd{TCODE} column to the value \cmd{37}, and making an 
appropriate notation like \emph{Section 24-25-26-27} in the 
\cmd{REMARKS} column.\\
\FIXME{Perhaps the \cmd{SCODE} column should accept an array of sections?}

\end{minipage}
}
\end{center}


%% ======================== Packages ==============================
%% ======================== Packages ==============================
\newpage
\section{Packages}


%% ======================== Functions =============================
%% ======================== Functions =============================
\newpage
\section{Functions}


%% ======================== Variables =============================
%% ======================== Variables =============================
\newpage
\section{Variables}

%% ======================= Flow Control ===========================
%% ======================= Flow Control ===========================
\newpage
\section{Flow Control}
Flow control of code using conditionals like \texttt{for},
\texttt{if}, \texttt{else}, \texttt{switch},
and defer.\footnote{``Flow control statements: for, if, else,
switch and defer.  Learn how to control the flow of your code
with conditionals, loops, switches and defers.''\\\href{https://tour.golang.org/list}{https://tour.golang.org/list}}

%% ======================== Examples =============================
%% ======================== Examples =============================
\newpage
\section{Examples}


%% ====================== Verbatim ==========================
%% ====================== Verbatim ==========================
\clearpage
\newpage
\subsection{Verbatim}

``The verbatim environment is a paragraph-making environment that gets
\LaTeX\ to print exactly what you type in. It turns \LaTeX\ into a typewriter
with carriage returns and blanks having the same effect that they would
on a typewriter.''\footnotemark

\begin{verbatim}
\begin{verbatim}
    text
\end{verbatim\}
\end{verbatim}


%% ========= Figure formatting with verbatim ===============
%% ========= Figure formatting with verbatim ===============
\noindent\begin{bf}Figure formatting with verbatim\end{bf}\\
%%
%% Making Figures in LaTeX
%% The [htb] part above advises LaTeX to put the figure "here"
%% or at the "top" or " bottom" of the page, with that order of preference.
%% http://www.sci.utah.edu/~macleod/latex/latex-figures.html
%%
\noindent The following figure leverages verbatim for proper formatting:
\vspace{-5mm}
\begin{center}
\begin{figure}[htb]
\begin{quote}
\small
\begin{Verbatim}[frame=none]
gis/raster/
  dnr/
    map_library/
    plats/
      SP/YYYYMMDD/*.pdf               # indexed
      HI/YYYYMMDD/*.pdf               # Indexed
      ASLS/YYYYMMDD/*.pdf             # Indexed
    recorded-plats/
      YYYYMMDD/*.pdf
  blm/
    easements_17b/YYYYMMDD/*.pdf      # indexed
    mtp/YYYYMMDD/*.pdf                # non-indexed
    usrs/YYYYMMDD/*.pdf               # indexed
    usrs-notes/YYYYMMDD/*.pdf         # indexed
    uss/YYYYMMDD/*.pdf                # indexed
    uss-notes/YYYYMMDD/*.pdf          # indexed
    usms/YYYYMMDD/*.pdf               # indexed
    usms-notes/YYYYMMDD/*.pdf         # indexed
  usgs/
    drg/
      collared/ 
        250K/
        63K/
        25K/
        24/
      decollared/
      tools/
      missing\_data/
    dem/
    doq/
    topo/
\end{Verbatim}
\normalsize
\end{quote}
\caption{File and Directory Structure}
\label{fig:dirlayout}
\end{figure}
\end{center}


%% ======================== Appendix =============================
%% ======================== Appendix =============================
%%
%% This will add a standard non-numbered Appendix to the document.
%% The next section Appendix chnages secnumdepth such that the Appendix
%% is not numbered but still displayed in the table of contents (TOC).
%%
%% Adding unnumbered sections to TOC
%% http://tex.stackexchange.com/questions/11668/adding-unnumbered-sections-to-toc
%% 
%% \section*{Appendix}
%% Need content here.
%%
\setcounter{secnumdepth}{0}
\section{Appendix}


%% ========================== Links ==============================
%% ========================== Links ==============================
\subsection{Links}


%% ============= A Tour of Go ===============
A Tour of Go\\
The tour covers the most important features of the language,
mainly: Basics (Packages, variables, functions, flow control
statements: for, if, else, switch, defer, more types:
structs, slices, and maps), Methods and interfaces, and
Concurrency.\\
\href{https://tour.golang.org/welcome/}{https://tour.golang.org/welcome/1}
\\

%% ============= Package Types ===============
\noindent Package types\\
\texttt{import "go/types"}\\
Package types declares the data types and implements
the algorithms for type-checking of Go packages. Use
\texttt{Config.Check} to invoke the type checker for
a package.  Alternatively, create a new type checked
with \texttt{NewChecker} and invoke it incrementally
by calling \texttt{Checker.Files}.\\
The Go Programming Language\\
\href{https://golang.org/pkg/go/types/}{https://golang.org/pkg/go/types/}


\end{document}

%% Local Variables:
%% fill-column: 78
%% mode: auto-fill
%% compile-command: "make"
%% End:
